%&latex
\documentclass[11pt]{article}
%%%%%%%%%%%%%%%%%%%%%%%%%%%%%%%%%%%%%%%%%%%%%%%%%%%%%%%%%%%%%%%%%%%%%%%%%
%
% checksum = "60469 113 371 3478"
%
% This document has been altered if the CRC above does not 
% match using Robert M. Solovay's utility. 
%
%%%%%%%%%%%%%%%%%%%%%%%%%%%%%%%%%%%%%%%%%%%%%%%%%%%%%%%%%%%%%%%%%%%%%%%%%
%
% A hack to give the page an European/decent look and fit the text on 
% the page...
%
% Feel free to use it but it works with US Letter and 11pt only!
% Who said that Colombia uses ISO/DIN standards? 
% Only on paper, which is usually US Legal in the government
% business. :)
%
\setlength{\baselineskip}{13.6pt}
\setlength{\evensidemargin}{17.34pt}
\setlength{\footskip}{47.6pt}
\setlength{\headheight}{17.0pt}
\setlength{\headsep}{20.4pt}
\setlength{\oddsidemargin}{17.34pt}
\setlength{\paperheight}{845.0pt}
\setlength{\paperwidth}{598.0pt}
\setlength{\parindent}{0.0pt}
\setlength{\parskip}{0.40\baselineskip} % hack for \parindent
\setlength{\textheight}{596.0pt}
\setlength{\textwidth}{418.3pt}
\setlength{\topmargin}{-25.2pt}
\setlength{\topskip}{11.0pt}
\frenchspacing % I ain't using no typewriter!
\nofiles % What is it to ouput anyway?
%
\def\OzTeX{O\kern-.03em z\kern-.15em\TeX}
\def\marvo{\textsc{Marvosym}}
\def\thefont{\textsc{Martin Vogels Symbole}}
\def\atm{\textsc{Adobe Type Manager\texttrademark\ Deluxe}}
\def\suitc{\textsc{Suitcase\texttrademark\ \textnormal{2.1 or 3.x}}}
\def\juggler{\textsc{Master Juggler\texttrademark}}
\def\macos{\textsc{M\textnormal{ac}OS}}
\def\qdraw{\textsc{QuickDraw\texttrademark}}
\def\pfa{\textsc{pfa}}
\def\ascii{\textsc{ascii}}
\def\lwfn{\textsc{lwfn}}
\def\tfm{\textsc{tfm}}
\def\dvips{\textsc{dvips}}
\def\lw{LaserWriter\texttrademark}
\def\yy{\textsc{y\&y}}
\def\html{\textsc{html}}
\def\dvi{\textsc{dvi}}
\def\PS1{\textsc{PS T\textnormal{ype 1}}}
%
\begin{document}
\title{\thefont \\ for the \macos}
\author{P. Alejandro L\'opez-Valencia}
\date{August, 1998}
\maketitle
\thispagestyle{empty}

This is a \macos\ release of the free font \thefont\ \textcopyright 
1998 Martin Vogel,  
\verb|<http://www.fh-bochum.de/fb1/vogel/marvosym.html>|.

My contributions are the creation of a bitmap suitcase to make the 
font available to the \macos, be it installed in the Fonts folder or 
using a font manager such as \atm, \suitc\ or \juggler; and the 
lossless conversion of the \PS1\ outline from Printer File \ascii, 
\pfa, to \lw\ Font, \lwfn\, format.
The font outline was converted from TrueType to 
\PS1\ and custom reencoded by Thomas Henlich 
\verb|(thenlich@rcs.urz.tu-dresden.de)|, and later hand hinted by \yy,
\verb|<http://www.yandy.com/>|.

Included in this package are:

\begin{itemize}

\item The  \LaTeX\ style and ``catalog'' files 
contributed by Thomas Henlich, and available from 
\verb|CTAN:fonts/psfonts/marvosym/|.

\item The original documentation in \html\ format.

\item Configuration files for \OzTeX\ which use the system 
installed \PS1\  outline:

\begin{itemize}

\item Configuration file to use \marvo\ for \dvi\ previewing and 
\qdraw\ printing.

\item \tfm\ file.

\item \dvips\ map and config files for \marvo.

\end{itemize}

\item The original \yy\ fonts in \pfa\ format, in 
case you prefer to use them with \dvips. This is left as an 
exercise to the reader\dots

\end{itemize}

\vspace{\baselineskip}

P. Alejandro L\'opez-Valencia \\
\verb|palopez@usa.net| \\
Santa Fe de Bogot\'a, Colombia. \\
25\textsuperscript{th}\ August, 1998.
\end{document}
